\section{Diseño de los algoritmos}
\subsection{Test de primalidad}
Según el Teorema fundamental de la aritmética, todo entero positivo mayor que $1$1 es un número primo o bien un único producto de números primos, es decir, que un número natural es primo si y sólo si es divisible únicamente por sí mismo y por $1$.\\
Por lo tanto, la forma más sencilla de comprobar si un número es primo es comprobar si es divisible por cualquier número natural de $2$ a $n-1$.\\
\\
Sin embargo, no hace falta probar todos los números naturales hasta $n-1$. Al encontrar un divisor $p$ de $n$, en realidad estamos encontrando también el divisor $q$, ya que $n / p = q$. Pongamos que $n$ es el cuadrado de $p$, es decir, $n = p \times p \Rightarrow p = \sqrt{n}$. Como estamos probando números naturales de menor a mayor, cualquier número $q > p$ que también sea divisor de $n$ ya habría sido detectado anteriormente. Por lo tanto, no es necesario probar números superiores a $\sqrt{n}$. Como estamos trabajando con números naturales, aplicaremos la función suelo a la raíz cuadrada, ya que si $p \leq \sqrt{n} \rightarrow p \leq \left \lfloor \sqrt{n} \right \rfloor \leq \sqrt{n} \leq \left \lceil \sqrt{n} \right \rceil \Rightarrow p \leq \left \lfloor \sqrt{n} \right \rfloor$.\\
Otra optimización es, en el caso de que $n$ no sea par, saltarnos los números pares.\\
\\
Por lo tanto, el algoritmo diseñado para comprobar si un número natural $n$ es primo o compuesto es el siguiente:
\begin{enumerate}
    \item Si $n$ es divisible entre $2$, es compuesto.
    \item Por cada número natural \textit{impar} $p \in [3, \left \lfloor \sqrt{n} \right \rfloor]$:
    \begin{enumerate}
        \item Si $n$ es divisible por $p$, es compuesto.
    \end{enumerate}
    \item En cualquier otro caso, es primo.
\end{enumerate}

La implementación usada, en C++, se puede encontrar en \href{run:./src/primes/primes.hpp}{\texttt{src/primes/primes.hpp}} (función \texttt{is\_prime}).


\subsection{Calculo de Máximo Común Divisor}

El máximo común divisor de dos números naturales distintos de $0$, $a$ y $b$, es el mayor número natural que es divisor de ambos.\\
\\
Hemos usado dos algoritmos distintos para resolver este problema: descomposición en factores primos y el método de Euclides.

\subsubsection{Descomposición en factores primos}

Un número natural se puede descomponer en sus factores primos, es decir, se puede expresar como el producto de sus divisores primos elevados a algún número natural:

\begin{gather*}
    a = \prod p_n^{q_n} \\ 
    q_n \in \mathbb{N}^+,\ p_n \in \mathbb{N}_p
\end{gather*}

Siendo $p_n^{q_n}$ un factor primo de $a$.\\
\\
El MCD es el producto de los factores comunes elevados al menor de los exponentes.
\begin{gather*}
    MCD(a, b) = \prod min(p_{a,n}^{q_{a,n}},\ p_{b,n}^{q_{b,n}}) \\ 
    q_n \in \mathbb{N}^+,\ p_n \in \mathbb{N}_p
\end{gather*}
\\
Por lo tanto, el algoritmo es el siguiente:
\begin{enumerate}
    \item Si $a=b$, el resultado es $a$.
    \item Factorizar $a$ y $b$.
    \item Guardar $2$ en el resultado.
    \item El número tenga el menor número de factores será $m$, el otro será $n$.
    \item Por cada factor de $m$:
    \begin{enumerate}
        \item Si $n$ también cuenta con ese factor, añadir (multiplicar) el factor con el mínimo exponente al resultado.
    \end{enumerate}
\end{enumerate}

El algoritmo de factorización de un número $n$ es:
\begin{enumerate}
    \item Guardar $2$ en $i$.
    \item Mientras $n$ sea mayor que $1$:
    \begin{enumerate}
        \item Si $n$ es divisible entre $i$:
        \begin{enumerate}
            \item Si $i$ está en la lista de factores, sumar $1$ a su exponente. En caso contrario, insertar $i$ en la lista de factores, con exponente $1$.
            \item Dividir $n$ entre $i$ y guardarlo en $n$.
        \end{enumerate}
        \item En caso contrario aumentar $i$ en 1.
    \end{enumerate}
\end{enumerate}


\subsubsection{Método de Euclides}

% TODO: explain stuff

\begin{enumerate}
    \item Si $a=b$, el resultado es $a$.
    \item Mientras $b$ sea mayor que $0$:
    \begin{enumerate}
        \item Calcular el resto entre $a$ y $b$
        \item Guardar $b$ en $a$ y el resto en $b$
    \end{enumerate}
\end{enumerate}

