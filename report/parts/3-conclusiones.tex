\part{Conclusiones}

\section{Desafios}
Durante el desarrollo de la práctica hemos encontrado diversos desafios, uno de los más notables se trata de la disonancia entre los resultados esperados y los obtenidos al medir el rendimiento en tiempo de ejcución del algoritmo. Como se explica en el aparado previo, se esperaba una distribución lineal, pero se obtuvo una logarítmica. Tras revisarlo, esto se debe a que el gráfico muestra una escala con al misma distancia entre puntos en el eje X, al tratase este de números de digitos, la distancia entre digitos debería ser cada vez mayor puesto que por ejemplo entre los numeros 1000 y 100, hay mayor cantidad de valores posibles que entre 100 y 10. Si esto fuese así si que se observaría una gráfica lineal.

\section{Conclusiones Generales}
En general el desarrollo de esta práctica nos ha resultado útil para poner en práctica ciertos conocimientos como el uso de matplotlib parapresentar de manera clara y visual visual los resultados obtenidos durante el desarrollo, así como el empleo de dataframes que en particular para esta esta prátcia han resultado de gran utilidad puesto que estas estructuras permiten trabajar de manera práctica con datos ya sea para representarlos o por ejemplo para guardarlos, como se ha hecho, en formato CSV.



\newpage
