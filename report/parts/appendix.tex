\appendix
\section{Estructura de los archivos de la práctica}

Los archivos se estructuran de la siguiente forma, por carpetas:
\begin{itemize}
    \item \texttt{data/}: Incluye los archivos CSV con los datos de los diferentes tests realizados.
    \item \texttt{src/}: Incluyen los archivos fuentes del código usado.
    \begin{itemize}
        \item \texttt{gcd/}: Archivos fuente de los algoritmos de Máximo Común Divisor.
        \begin{itemize}
            \item \texttt{gcd.hpp}: Archivo con la implementación de los algoritmos.
            \item \texttt{gcdlib.cpp}: Archivo de configuración del módulo para pybind11.
            \item \texttt{test.cpp}: Archivo fuente en C++ usado para la realización de pruebas durante el desarrollo.
        \end{itemize}
        \item \texttt{primes/}: Archivos fuente del algoritmo de test de primalidad.
        \begin{itemize}
            \item \texttt{primes.hpp}: Archivo con la implementación del algoritmo.
            \item \texttt{primeslib.cpp}: Archivo de configuración del módulo para pybind11.
            \item \texttt{test.cpp}: Archivo fuente en C++ usado para la realización de pruebas durante el desarrollo.
        \end{itemize}
        \item \texttt{test.py}: \textit{Script} de Python para la generación de las pruebas y gráficos.
    \end{itemize}
    \item \texttt{README.md}: Documento con instrucciones de compilación.
\end{itemize}

También se cuenta con archivos \texttt{CMakeLists.txt}, los cuales fueron usados para compilar los distintos módulos.
