\part{Desarrollo}


\section{Diseño de los algoritmos}
\subsection{Test de primalidad}
Según el Teorema fundamental de la aritmética, todo entero positivo mayor que $1$1 es un número primo o bien un único producto de números primos, es decir, que un número natural es primo si y sólo si es divisible únicamente por sí mismo y por $1$.\\
Por lo tanto, la forma más sencilla de comprobar si un número es primo es comprobar si es divisible por cualquier número natural de $2$ a $n-1$.\\
\\
Sin embargo, no hace falta probar todos los números naturales hasta $n-1$. Al encontrar un divisor $p$ de $n$, en realidad estamos encontrando también el divisor $q$, ya que $n / p = q$. Pongamos que $n$ es el cuadrado de $p$, es decir, $n = p \times p \Rightarrow p = \sqrt{n}$. Como estamos probando números naturales de menor a mayor, cualquier número $q > p$ que también sea divisor de $n$ ya habría sido detectado anteriormente. Por lo tanto, no es necesario probar números superiores a $\sqrt{n}$. Como estamos trabajando con números naturales, aplicaremos la función suelo a la raíz cuadrada, ya que si $p \leq \sqrt{n} \rightarrow p \leq \left \lfloor \sqrt{n} \right \rfloor \leq \sqrt{n} \leq \left \lceil \sqrt{n} \right \rceil \Rightarrow p \leq \left \lfloor \sqrt{n} \right \rfloor$.\\
Otra optimización es, en el caso de que $n$ no sea par, saltarnos los números pares.\\
\\
Por lo tanto, el algoritmo diseñado para comprobar si un número natural $n$ es primo o compuesto es el siguiente:
\begin{enumerate}
    \item Si $n$ es divisible entre $2$, es compuesto.
    \item Por cada número natural \textit{impar} $p \in [3, \left \lfloor \sqrt{n} \right \rfloor]$:
    \begin{enumerate}
        \item Si $n$ es divisible por $p$, es compuesto.
    \end{enumerate}
    \item En cualquier otro caso, es primo.
\end{enumerate}

La implementación usada, en C++, se puede encontrar en \href{run:./src/primes/primes.hpp}{\texttt{src/primes/primes.hpp}} (función \texttt{is\_prime}).


\section{Evaluación analítica}
\subsection{Test de primalidad}

Teniendo en cuenta el algoritmo descrito anteriormente, podemos calcular analíticamente el coste del mismo.\\
Asumiendo cualquier coste no dependiente del tamaño del número de entrada $n$ como constante, el coste del algoritmo viene derivado del paso 2, el cual es un bucle que se repite en el peor de los casos $(\left \lfloor \sqrt{n} \right \rfloor - 3)/2$ veces, por lo que podemos decir que la complejidad temporal es:
\begin{equation}
    T(n) = \frac{\left \lfloor \sqrt{n} \right \rfloor - 3}{2}
\end{equation}

La memoria es independiente del tamaño del problema, por lo que la complejidad temporal es:
\begin{equation}
    S(n) = 1
\end{equation}



\section{Evaluación empírica}
\subsection{Test de primalidad}


\newpage