\section{Evaluación analítica}
\subsection{Test de primalidad}

Teniendo en cuenta el algoritmo descrito anteriormente, podemos calcular analíticamente el coste del mismo.\\
Asumiendo cualquier coste no dependiente del tamaño del número de entrada $n$ como constante, el coste del algoritmo viene derivado del paso 2, el cual es un bucle que se repite en el peor de los casos $(\left \lfloor \sqrt{n} \right \rfloor - 3)/2$ veces, por lo que podemos decir que la complejidad temporal es:
\begin{equation}
    T(n) = \frac{\left \lfloor \sqrt{n} \right \rfloor - 3}{2}
\end{equation}

La memoria es independiente del tamaño del problema, por lo que la complejidad temporal es:
\begin{equation}
    S(n) = 1
\end{equation}